\section*{Conclusion} 


La reconstruction du flux de particules, qui améliore nettement les performances de CMS et permet de tirer le meilleur parti des données collectées, est maintenant utilisée dans la quasi-totalité des mesures réalisées par la collaboration. 
Les états finals considérés dans la recherche de $\PH \to \tau \tau$ font intervenir l'ensemble des objets de physique de CMS, et le PF influe donc particulièrement sur les performances de cette analyse. 
Pour quantifier les améliorations apportées par le PF sur cette analyse, il aurait fallu refaire l'analyse avec les objets traditionnels.
Cependant, bon nombre de ces objets ne sont plus maintenus par la collaboration; ils ne sont plus présents dans la reconstruction standard, et les diverses mesures nécessaires pour une analyse basée sur ces objets (incertitudes systématiques, facteurs de correction pour la simulation) ne sont plus réalisées. 
De plus, il aurait fallu revoir complètement la stratégie d'analyse pour l'adapter à ces objets moins performants. 
Il est néanmoins possible de se faire une idée qualitative des répercussions les plus importantes du PF sur cette analyse. 
L'efficacité d'identification des $\tauh$, tout d'abord, est améliorée de plus d'un facteur deux pour une même probabilité d'identifier un jet comme $\tauh$. 
Dans le canal complètement hadronique $\PH \to \tauh \tauh$, cette amélioration se traduit par une diminution d'un facteur quatre du bruit de fond le plus important provenant de la production QCD d'événements multijets. 
L'analyse aurait pu être envisagée dans ce canal avec les $\tauh$ calorimétriques traditionnels
mais il aurait alors fallu augmenter le seuil sur l'impulsion transverse du boson de Higgs candidat, $p_{\rm T}^{\tau \tau}$, 
au prix d'une réduction de l'efficacité de sélection du signal.   
Dans les canaux $\mu \tauh$ et $\Pe \tauh$, le bruit de fond $\PW$+jets joue un rôle majeur, 
et ce bruit de fond est supprimé en rejetant les événements avec $m_{\rm T}>30\,\GeV$. 
Le PF améliore grandement la résolution sur \vecptmiss (sur sa norme comme sur son azimuth), 
et permet de conserver ses améliorations avec niveau d'empilement élevé. 
Revenir à une reconstruction calorimétrique de \vecptmiss étalerait les distributions en $m_{\rm T}$ 
du signal et du bruit de fond $\PW$+jets, et les rendrait difficilement séparables. 
La résolution sur \vecptmiss domine également la reconstruction de la masse du système $\tau \tau$ par le {\sc SVFIT}.
Pour $m_{\rm H} = 125\,\GeV$, la distribution en $m_{\tau\tau}$ du signal se trouve sur le front descendant de celle du bruit de fond 
$\cPZ \to \tau \tau$, et les deux pics sont à peine résolus, même avec la PF \vecptmiss.
Avec la \vecptmiss traditionnelle, la séparation entre ces deux pics deviendrait extrêmement difficile. 
Il se pourrait alors que la recherche du signal ne puisse plus se faire dans la distribution de $m_{\tau\tau}$,  
ce qui compliquerait fortement la mesure de la masse de la résonance détectée. 



premiere mise en evidence de H->tau tau et donc du couplage du boson de Higgs a la matiere
incertitude sur $\kappa_F$ ($\kappa_\tau$)

projections run II - mentionner 2 thesards. 
projections HL-LHC (se renseigner là-dessus)

Mesure des couplages : FCC (nommer ILC ou pas?) 

FCC 
LEP3/TLEP 
design study 
fcc software convener
papas 

outlook PF -- comment integrer ca? 
run 2, HL-LHC, FCC ee et hh 

\begin{itemize}
\item Comparer target PTDR aux resultats obtenus ici. le faire pour la soutenance, mais ne pas le mentionner. 
\item questions H->tau tau 
\item evolution sigma(H->tautau) vs mH
\item revoir sigma H, BR H
\item  Higgs CP measurement H->tau tau : contact Yuta / Emmanuelle? 
\item   MSSM H->tautau relire papiers
\item   relire combination papers (CMS, CMS+ATLAS) - hypothèses pour la mesure des couplages? 
\item   lire papiers H->tautau atlas
\item PF 
\item   ILD, calice, etc.
\item  ALEPH 
\item   Apprendre caracteristiques des differents detecteurs
\end{itemize}
