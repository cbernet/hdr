\section*{Résumé}

CMS est, avec ATLAS, l'une des deux grandes expériences généralistes installées au LHC, le grand collisionneur de hadrons du CERN.
Le détecteur CMS est un cylindre de 25\,m de long et 15\,m de diamètre, constitué de sous-détecteurs arrangés en couches successives. 
Chacun de ces sous-détecteurs fut pensé en vue de la mesure d'une observable particulière pour l'étude des événements de collision proton-proton du LHC. 
Ainsi, le détecteur de traces central, qui permet la reconstruction de la trajectoire des particules chargées, fut principalement conçu dans le but d'identifier les jets issus de quarks b et les désintégrations hadroniques du lepton $\tau$. 
Le calorimètre électromagnétique fut créé spécifiquement pour la détection des photons et des électrons énergétiques, et le calorimètre hadronique pour celle des jets et de l'énergie transverse manquante. 
Enfin, les détecteurs situés à la périphérie de l'appareillage ont pour but l'identification et la reconstruction des muons, 
seules particules détectables à parvenir jusqu'à eux. 
Plutôt que de considérer les sous-détecteurs indépendamment, l'algorithme de reconstruction du flux de particules (PF) combine les informations de tous les sous-détecteurs pour reconstruire et identifier de manière optimale toutes les particules stables de l'état final de la collision: electrons, muons, photons, hadrons chargés, et hadrons neutres. 
Cette approche novatrice de la reconstruction, développée et utilisée avec succès en collision $\rm e^+e^-$, 
est présentée pour la première fois en collision proton-proton, pour CMS.  
La liste des particules produites par le PF constitue une description globale de l'événement de collision Elle est utilisée pour reconstruire avec une précision accrue toutes les observables utilisées dans les mesures physiques. 
Ainsi, à titre d'example, les jets de particules et l'énergie transverse manquante sont reconstruits avec une résolution en énergie et en direction améliorée d'un facteur deux par rapport à ceux obtenus avec les techniques de reconstruction traditionnelles, basées sur les calorimètres. 
Le PF permet une reconstruction précise toutes les particules, qu'elles proviennent de la collision proton-proton principale ayant donné lieu au déclenchement de l'acquisition, ou des collisions proton-proton parasites se produisant simultanément, lors du même croisement de paquets de protons. 
Cette description fine de l'état final permet l'identification et la suppression de la majeure partie des particules provenant de collisions parasites. 

L'ensemble des observables obtenues grâce au PF est utilisé pour la recherche d'un boson de Higgs du modèle standard se désintégrant en paire de $\tau$ dans les données collectées par CMS durant le run I du LHC, en 2011 et 2012. 
Ce mode de désintégration est le seul permettant d'accéder au couplage du boson de Higgs aux leptons au LHC du fait de la masse élevée du $\tau$. 
Plus généralement, il est de loin le plus prometteur pour s'assurer de l'existence même des couplages de Yukawa, 
c'est à dire de la capacité du boson de Higgs d'interagir avec la matière, et donc de lui donner sa masse. 
Dans 65\% des cas, le $\tau$ se désintègre de manière hadronique pour produire un neutrino et des hadrons chargés (généralement au nombre de un ou de trois), parfois accompagnés de $\pi^0$ qui se désintègrent immédiatement en deux photons. Sinon, la désintégration leptonique du $\tau$ produit un électron ou un muon, ainsi que deux neutrinos.
La recherche est réalisée pour tous les canaux, $\PH \to \tau \tau = \mu \tauh$, $\Pe\tauh$, $\tauh \tauh$,  $\Pe \mu$, $\Pe \Pe$, $\mu \mu$, où $\tauh$ dénote les produits visibles de la désintégration hadronique.
Dans chacun de ces canaux, la contribution du bruit de fond, extrêmement importante, est fortement réduite par une sélection et une catégorisation d'événements adéquate. 
La catégorisation vise, en particulier, les différents modes de production du boson de Higgs grâce à des critères appliqués aux objets présents dans l'état final en plus de la paire de $\tau$: fusion gluon-gluon (1 jet), fusion de bosons vecteurs (2 jets), production associée à un boson $\PW$ ou $\cPZ$ (électrons ou muons).
Le niveau de bruit reste cependant important après cette sélection, avec un rapport signal sur bruit de l'ordre de 10\%. 
Les contributions résiduelles des sources de bruit de fond sont ensuite estimées avec précision à partir d'échantillons de données de contrôle. 
On observe un excès d'événements par rapport à l'hypothèse d'un modèle standard sans boson de Higgs. 
Cet excès est supérieur à trois déviations standard pour des hypothèses de masse du boson de Higgs comprises entre $m_{\rm H} = 115$ et 130\,\GeV. 
Pour $m_{\rm H} = 125\,\GeV$, le produit de la section efficace par le rapport d'embranchement pour le signal observé est mesuré à $0.78 \pm 21$ fois la valeur attendue pour le modèle standard. 
La masse de cette résonance, en supposant qu'il s'agit bien de $\PH \to \tau \tau$, est $m_{\rm H} = 122 \pm 7\,\GeV$. 
Ces résultats constituent une mise en évidence du couplage entre le $\tau$ et le boson de Higgs découvert à une masse de 125\,\GeV en 2012 par les collaborations ATLAS et CMS. 
